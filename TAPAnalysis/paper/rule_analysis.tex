\documentclass{sig-alternate-05-2015}

\usepackage{listings}
\usepackage{textcomp}
\lstset{basicstyle=\ttfamily}
\usepackage{xcolor}
\newcommand\todo[1]{\textbf{\textcolor{red}{#1}}}

\toappear{}
\begin{document}

% Copyright
\setcopyright{acmcopyright}
%\setcopyright{acmlicensed}
%\setcopyright{rightsretained}
%\setcopyright{usgov}
%\setcopyright{usgovmixed}
%\setcopyright{cagov}
%\setcopyright{cagovmixed}


% DOI
\doi{10.475/123_4}

% ISBN
\isbn{123-4567-24-567/08/06}

%Conference
\conferenceinfo{PLDI '13}{June 16--19, 2013, Seattle, WA, USA}

\acmPrice{\$15.00}

%
% --- Author Metadata here ---
\conferenceinfo{WOODSTOCK}{'97 El Paso, Texas USA}
%\CopyrightYear{2007} % Allows default copyright year (20XX) to be over-ridden - IF NEED BE.
%\crdata{0-12345-67-8/90/01}  % Allows default copyright data (0-89791-88-6/97/05) to be over-ridden - IF NEED BE.
% --- End of Author Metadata ---

\title{Automatic Trigger Generation for Rule-based Smart Homes}
%
% You need the command \numberofauthors to handle the 'placement
% and alignment' of the authors beneath the title.
%
% For aesthetic reasons, we recommend 'three authors at a time'
% i.e. three 'name/affiliation blocks' be placed beneath the title.
%
% NOTE: You are NOT restricted in how many 'rows' of
% "name/affiliations" may appear. We just ask that you restrict
% the number of 'columns' to three.
%
% Because of the available 'opening page real-estate'
% we ask you to refrain from putting more than six authors
% (two rows with three columns) beneath the article title.
% More than six makes the first-page appear very cluttered indeed.
%
% Use the \alignauthor commands to handle the names
% and affiliations for an 'aesthetic maximum' of six authors.
% Add names, affiliations, addresses for
% the seventh etc. author(s) as the argument for the
% \additionalauthors command.
% These 'additional authors' will be output/set for you
% without further effort on your part as the last section in
% the body of your article BEFORE References or any Appendices.

\numberofauthors{2} %  in this sample file, there are a *total*
% of EIGHT authors. SIX appear on the 'first-page' (for formatting
% reasons) and the remaining two appear in the \additionalauthors section.
%
\author{
% 1st. author
\alignauthor
Chandrakana Nandi\\
       \affaddr{University of Washington, Seattle}\\
       \email{cnandi@cs.washington.edu}
\alignauthor Michael D. Ernst\\
       \affaddr{University of Washington, Seattle}\\
       \email{mernst@cs.washington.edu}
}

\maketitle
\begin{abstract}
[\todo {Write abstract}]
\end{abstract}

\printccsdesc


\keywords{Security, Home automation, Trigger action programming, Static analysis}

\section{Introduction}
Most home automation frameworks have a major end-user component: they are the ones who decide what actions should be taken by what device under what conditions. It has been found that rule-based systems are one of the most practical solutions for enabling end-users to customize the behavior of their smart homes~\cite{practical-tap}. A rule-based language has two main components: actions to be executed when a rules fires and triggers that cause a rule to be fired. Such a programming paradigm is also called Trigger Action Programming (TAP). Recent studies~\cite{Huang},\cite{wild-tap} have shown that while TAP is the most commonly used and practical approach for home automation, end users often have difficulty in correctly interpreting the outcome of TAP and make errors in writing a program with the desired behavior. 

\section{Motivating Example}
Consider the following rule:
\begin{lstlisting}
rule "Away rule"
when
  Item State_Away changed 	
then
  if(State_Away.state == ON){
    if(State_Sleeping.state != OFF){
      postUpdate(State_Sleeping,OFF)
    }
  }
end
\end{lstlisting}
This rule is supposed to set the value of \texttt{State\_Away} to \texttt{ON} when everyone is away. It also sets \texttt{State\_Sleeping} to \texttt{OFF} if it currently \texttt{ON} so that both \texttt{State\_Away} and \texttt{State\_Sleeping} are not \texttt{ON} simultaneously. However, the trigger for this rule only contains \texttt{State\_Away}. As a result, this rule is only fired when \texttt{State\_Away} is changed by the end user. However, if the value of \texttt{State\_Sleeping} is changed instead, then the rule is not fired due to which, it allows both values to be set at the same time. This deviates from the expected behavior of the rule--it is no more clear whether the home inhabitants are away or inside sleeping. This is also a security vulnerability because if this rule triggers another rule for turning on the security alarm in the house and the security alarm is meant to be off when people are at home but on when they are away, then it could get deactivated if the value of \texttt{STATE\_SLEEPING} is set to \texttt{ON}. 

\section{Approach}
\section{Implementation}
\section{Experimental Evaluation}
\section{Related work}
\section{Conclusions}


\section{Acknowledgments}

\bibliographystyle{abbrv}
\bibliography{sigproc}  

\end{document}
